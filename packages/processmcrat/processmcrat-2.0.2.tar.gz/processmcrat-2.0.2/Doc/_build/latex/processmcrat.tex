%% Generated by Sphinx.
\def\sphinxdocclass{report}
\documentclass[letterpaper,10pt,english]{sphinxmanual}
\ifdefined\pdfpxdimen
   \let\sphinxpxdimen\pdfpxdimen\else\newdimen\sphinxpxdimen
\fi \sphinxpxdimen=.75bp\relax

\PassOptionsToPackage{warn}{textcomp}
\usepackage[utf8]{inputenc}
\ifdefined\DeclareUnicodeCharacter
% support both utf8 and utf8x syntaxes
  \ifdefined\DeclareUnicodeCharacterAsOptional
    \def\sphinxDUC#1{\DeclareUnicodeCharacter{"#1}}
  \else
    \let\sphinxDUC\DeclareUnicodeCharacter
  \fi
  \sphinxDUC{00A0}{\nobreakspace}
  \sphinxDUC{2500}{\sphinxunichar{2500}}
  \sphinxDUC{2502}{\sphinxunichar{2502}}
  \sphinxDUC{2514}{\sphinxunichar{2514}}
  \sphinxDUC{251C}{\sphinxunichar{251C}}
  \sphinxDUC{2572}{\textbackslash}
\fi
\usepackage{cmap}
\usepackage[T1]{fontenc}
\usepackage{amsmath,amssymb,amstext}
\usepackage{babel}



\usepackage{times}
\expandafter\ifx\csname T@LGR\endcsname\relax
\else
% LGR was declared as font encoding
  \substitutefont{LGR}{\rmdefault}{cmr}
  \substitutefont{LGR}{\sfdefault}{cmss}
  \substitutefont{LGR}{\ttdefault}{cmtt}
\fi
\expandafter\ifx\csname T@X2\endcsname\relax
  \expandafter\ifx\csname T@T2A\endcsname\relax
  \else
  % T2A was declared as font encoding
    \substitutefont{T2A}{\rmdefault}{cmr}
    \substitutefont{T2A}{\sfdefault}{cmss}
    \substitutefont{T2A}{\ttdefault}{cmtt}
  \fi
\else
% X2 was declared as font encoding
  \substitutefont{X2}{\rmdefault}{cmr}
  \substitutefont{X2}{\sfdefault}{cmss}
  \substitutefont{X2}{\ttdefault}{cmtt}
\fi


\usepackage[Bjarne]{fncychap}
\usepackage{sphinx}

\fvset{fontsize=\small}
\usepackage{geometry}

% Include hyperref last.
\usepackage{hyperref}
% Fix anchor placement for figures with captions.
\usepackage{hypcap}% it must be loaded after hyperref.
% Set up styles of URL: it should be placed after hyperref.
\urlstyle{same}
\addto\captionsenglish{\renewcommand{\contentsname}{Contents:}}

\usepackage{sphinxmessages}
\setcounter{tocdepth}{1}



\title{ProcessMCRaT}
\date{Apr 07, 2021}
\release{}
\author{Tyler Parsotan}
\newcommand{\sphinxlogo}{\vbox{}}
\renewcommand{\releasename}{}
\makeindex
\begin{document}

\pagestyle{empty}
\sphinxmaketitle
\pagestyle{plain}
\sphinxtableofcontents
\pagestyle{normal}
\phantomsection\label{\detokenize{index::doc}}



\chapter{mclib.py}
\label{\detokenize{mclib:module-mclib}}\label{\detokenize{mclib:mclib-py}}\label{\detokenize{mclib::doc}}\index{mclib (module)@\spxentry{mclib}\spxextra{module}}\index{Lorentz\_Boost() (in module mclib)@\spxentry{Lorentz\_Boost()}\spxextra{in module mclib}}

\begin{fulllineitems}
\phantomsection\label{\detokenize{mclib:mclib.Lorentz_Boost}}\pysiglinewithargsret{\sphinxcode{\sphinxupquote{mclib.}}\sphinxbfcode{\sphinxupquote{Lorentz\_Boost}}}{\emph{boost}, \emph{P\_ph}}{}
This routine performs a Lorentz boost of a 4-momentum. The boost is specified with a 3-vel.
\begin{quote}\begin{description}
\item[{Parameters}] \leavevmode\begin{itemize}
\item {} 
\sphinxstyleliteralstrong{\sphinxupquote{boost}} \textendash{} The velocity vector of the frame that the photon will be boosted into.

\item {} 
\sphinxstyleliteralstrong{\sphinxupquote{P\_ph}} \textendash{} The photon 4 momentum that will be boosted into the desired frame of reference.

\end{itemize}

\item[{Returns}] \leavevmode
the boosted photon 4 mometum

\end{description}\end{quote}

\end{fulllineitems}

\index{event3D() (in module mclib)@\spxentry{event3D()}\spxextra{in module mclib}}

\begin{fulllineitems}
\phantomsection\label{\detokenize{mclib:mclib.event3D}}\pysiglinewithargsret{\sphinxcode{\sphinxupquote{mclib.}}\sphinxbfcode{\sphinxupquote{event3D}}}{\emph{r\_obs}, \emph{theta\_deg}, \emph{phi\_deg}, \emph{dtheta\_deg}, \emph{path}, \emph{lastfile}, \emph{sim\_type}, \emph{riken\_switch=False}}{}
Place holder function to conduct a synthetic observation of MCRaT simulated outflow using a 3D hydro simulationp.
\begin{quote}\begin{description}
\item[{Parameters}] \leavevmode\begin{itemize}
\item {} 
\sphinxstyleliteralstrong{\sphinxupquote{r\_obs}} \textendash{} 

\item {} 
\sphinxstyleliteralstrong{\sphinxupquote{theta\_deg}} \textendash{} 

\item {} 
\sphinxstyleliteralstrong{\sphinxupquote{phi\_deg}} \textendash{} 

\item {} 
\sphinxstyleliteralstrong{\sphinxupquote{dtheta\_deg}} \textendash{} 

\item {} 
\sphinxstyleliteralstrong{\sphinxupquote{path}} \textendash{} 

\item {} 
\sphinxstyleliteralstrong{\sphinxupquote{lastfile}} \textendash{} 

\item {} 
\sphinxstyleliteralstrong{\sphinxupquote{sim\_type}} \textendash{} 

\item {} 
\sphinxstyleliteralstrong{\sphinxupquote{riken\_switch}} \textendash{} 

\end{itemize}

\item[{Returns}] \leavevmode


\end{description}\end{quote}

\end{fulllineitems}

\index{event4() (in module mclib)@\spxentry{event4()}\spxextra{in module mclib}}

\begin{fulllineitems}
\phantomsection\label{\detokenize{mclib:mclib.event4}}\pysiglinewithargsret{\sphinxcode{\sphinxupquote{mclib.}}\sphinxbfcode{\sphinxupquote{event4}}}{\emph{r\_obs}, \emph{theta\_deg}, \emph{dtheta\_deg}, \emph{path}, \emph{lastfile}, \emph{sim\_type}, \emph{withintheta\_deg=False}, \emph{riken\_switch=False}, \emph{save\_event=True}}{}
Function to collect MCRaT photons in space that will be observed. legacy code from when MCRaT produced text files
as its output.
\begin{quote}\begin{description}
\item[{Parameters}] \leavevmode\begin{itemize}
\item {} 
\sphinxstyleliteralstrong{\sphinxupquote{r\_obs}} \textendash{} radius of where the detector will be placed, should be at a radius such that all the photons have
propagated past the detector by the end of the MCRaT simulation

\item {} 
\sphinxstyleliteralstrong{\sphinxupquote{theta\_deg}} \textendash{} observer viewing angle in degrees

\item {} 
\sphinxstyleliteralstrong{\sphinxupquote{dtheta\_deg}} \textendash{} delta theta of the observer viewieng angle for accepting photons, also in degrees

\item {} 
\sphinxstyleliteralstrong{\sphinxupquote{path}} \textendash{} path to the directory that holds ll the output MCRaT files, should be a string

\item {} 
\sphinxstyleliteralstrong{\sphinxupquote{lastfile}} \textendash{} the number of the last MCRaT output file, this should be an int

\item {} 
\sphinxstyleliteralstrong{\sphinxupquote{sim\_type}} \textendash{} A string that will be the name of the output file of this function

\item {} 
\sphinxstyleliteralstrong{\sphinxupquote{withintheta\_deg}} \textendash{} this is a depreciated key

\item {} 
\sphinxstyleliteralstrong{\sphinxupquote{riken\_switch}} \textendash{} this is a depreciated key

\item {} 
\sphinxstyleliteralstrong{\sphinxupquote{save\_event}} \textendash{} switch to create the event file that will be read in by other functions

\end{itemize}

\item[{Returns}] \leavevmode
returns photon’s that would be observed and thier energy, detector crossing time, and their weight

\end{description}\end{quote}

\end{fulllineitems}

\index{event\_h5() (in module mclib)@\spxentry{event\_h5()}\spxextra{in module mclib}}

\begin{fulllineitems}
\phantomsection\label{\detokenize{mclib:mclib.event_h5}}\pysiglinewithargsret{\sphinxcode{\sphinxupquote{mclib.}}\sphinxbfcode{\sphinxupquote{event\_h5}}}{\emph{r\_obs}, \emph{theta\_deg}, \emph{dtheta\_deg}, \emph{path}, \emph{lastfile}, \emph{sim\_type}, \emph{fps=5}, \emph{read\_comv=False}, \emph{read\_stokes=False}, \emph{read\_type=False}, \emph{save\_event\_file=True}, \emph{append=False}}{}
Function to collect MCRaT photons in space that will be observed. Saves event files in EVENT\_FILES/ directory that
must be created before calling this functionp.
\begin{quote}\begin{description}
\item[{Parameters}] \leavevmode\begin{itemize}
\item {} 
\sphinxstyleliteralstrong{\sphinxupquote{r\_obs}} \textendash{} radius of where the detector will be placed, should be at a radius such that all the photons have
propagated past the detector by the end of the MCRaT simulation

\item {} 
\sphinxstyleliteralstrong{\sphinxupquote{theta\_deg}} \textendash{} observer viewing angle in degrees

\item {} 
\sphinxstyleliteralstrong{\sphinxupquote{dtheta\_deg}} \textendash{} delta theta of the observer viewing angle for accepting photons, also in degrees

\item {} 
\sphinxstyleliteralstrong{\sphinxupquote{path}} \textendash{} path to the directory that holds ll the output MCRaT files, should be a string

\item {} 
\sphinxstyleliteralstrong{\sphinxupquote{lastfile}} \textendash{} the number of the last MCRaT output file, this should be an int

\item {} 
\sphinxstyleliteralstrong{\sphinxupquote{sim\_type}} \textendash{} A string that will be the name of the output file of this function

\item {} 
\sphinxstyleliteralstrong{\sphinxupquote{fps}} \textendash{} an int that represents the number fo frames per second in the hydro simulation used

\item {} 
\sphinxstyleliteralstrong{\sphinxupquote{read\_comv}} \textendash{} switch to denote of the MCRaT files contain the comoving photon 4 momenta

\item {} 
\sphinxstyleliteralstrong{\sphinxupquote{read\_stokes}} \textendash{} switch to denote if the MCRaT output files have the stokes parameters

\item {} 
\sphinxstyleliteralstrong{\sphinxupquote{save\_event\_file}} \textendash{} switch to determine if the function should save the event file or not

\item {} 
\sphinxstyleliteralstrong{\sphinxupquote{append}} \textendash{} switch to denote if the function should append the extracted data to a pre-existing event file

\end{itemize}

\item[{Returns}] \leavevmode
returns (N,8) array where N is the number of photons ‘detected’. The array contains:
the crossing time of the photon, the lab energy in keV, the weight, the index in the list of photons,
the stokes parameters and the comoving energy in keV

\end{description}\end{quote}

\end{fulllineitems}

\index{lcur() (in module mclib)@\spxentry{lcur()}\spxextra{in module mclib}}

\begin{fulllineitems}
\phantomsection\label{\detokenize{mclib:mclib.lcur}}\pysiglinewithargsret{\sphinxcode{\sphinxupquote{mclib.}}\sphinxbfcode{\sphinxupquote{lcur}}}{\emph{simid}, \emph{t}, \emph{units='erg/s'}, \emph{theta=1.0}, \emph{dtheta=1.0}, \emph{phi=0}, \emph{dphi=1}, \emph{sim\_dims=2}, \emph{h5=True}, \emph{photon\_type=None}, \emph{energy\_range=None}}{}
reads in the event file and bins photons in uniform time bins to create light curves
\begin{quote}\begin{description}
\item[{Parameters}] \leavevmode\begin{itemize}
\item {} 
\sphinxstyleliteralstrong{\sphinxupquote{simid}} \textendash{} a string of the event file base name of the event file created in event\_h5 (everything but the .evt)

\item {} 
\sphinxstyleliteralstrong{\sphinxupquote{t}} \textendash{} an array of time bin edges

\item {} 
\sphinxstyleliteralstrong{\sphinxupquote{units}} \textendash{} string specifying units, can be erg/s or cts/s

\item {} 
\sphinxstyleliteralstrong{\sphinxupquote{theta}} \textendash{} observer viewing angle in degrees

\item {} 
\sphinxstyleliteralstrong{\sphinxupquote{dtheta}} \textendash{} the size of the observer viewing angle bin in degrees (same as is specified in event\_h5 function)

\item {} 
\sphinxstyleliteralstrong{\sphinxupquote{phi}} \textendash{} azimuthal angle for mock observer in degrees (only for 3D simulations, not fully supported)

\item {} 
\sphinxstyleliteralstrong{\sphinxupquote{dphi}} \textendash{} the size of the observer azimuthal viewing angle bin in degrees (only for 3D simulations, not fully
supported)

\item {} 
\sphinxstyleliteralstrong{\sphinxupquote{sim\_dims}} \textendash{} The number of dimensions of the hydro simulation used

\item {} 
\sphinxstyleliteralstrong{\sphinxupquote{h5}} \textendash{} specify if the format of the MCRaT output files is hdf5 files or not (denotes if using an old format or
a newer format of saving files)

\item {} 
\sphinxstyleliteralstrong{\sphinxupquote{photon\_type}} \textendash{} can be set to ‘s’, ‘i’, or left as None in order to select thermal synchrotron photons, injected photons, or all the photons in the simulation for analysis

\item {} 
\sphinxstyleliteralstrong{\sphinxupquote{energy\_range}} \textendash{} has units of keV, can be left as None to choose photons of all energy ranges for analysis or it can be set to an array with {[}min energy, max energy{]} e.g. {[}1, 10{]} for 1 to 10 keV (limits inclusive)

\end{itemize}

\item[{Returns}] \leavevmode
returns the time binned quantities of luminosity, luminosity error, number of photons in each bin, the
average eenrgy of photons in each bin, polarization, the stokes parameters, polarization error and
polarization angle

\end{description}\end{quote}

\end{fulllineitems}

\index{lcur\_old() (in module mclib)@\spxentry{lcur\_old()}\spxextra{in module mclib}}

\begin{fulllineitems}
\phantomsection\label{\detokenize{mclib:mclib.lcur_old}}\pysiglinewithargsret{\sphinxcode{\sphinxupquote{mclib.}}\sphinxbfcode{\sphinxupquote{lcur\_old}}}{\emph{simid}, \emph{t}, \emph{units='erg/s'}, \emph{theta=1.0}, \emph{dtheta=1.0}, \emph{phi=0}, \emph{dphi=1}, \emph{sim\_dims=2}, \emph{h5=True}, \emph{photon\_type=None}, \emph{energy\_range=None}}{}
reads in the event file and bins photons in uniform time bins to create light curves
\begin{quote}\begin{description}
\item[{Parameters}] \leavevmode\begin{itemize}
\item {} 
\sphinxstyleliteralstrong{\sphinxupquote{simid}} \textendash{} a string of the event file base name of the event file created in event\_h5 (everything but the .evt)

\item {} 
\sphinxstyleliteralstrong{\sphinxupquote{t}} \textendash{} an array of time bin edges

\item {} 
\sphinxstyleliteralstrong{\sphinxupquote{units}} \textendash{} string specifying units, can be erg/s or cts/s

\item {} 
\sphinxstyleliteralstrong{\sphinxupquote{theta}} \textendash{} observer viewing angle in degrees

\item {} 
\sphinxstyleliteralstrong{\sphinxupquote{dtheta}} \textendash{} the size of the observer viewing angle bin in degrees (same as is specified in event\_h5 function)

\item {} 
\sphinxstyleliteralstrong{\sphinxupquote{phi}} \textendash{} azimuthal angle for mock observer in degrees (only for 3D simulations, not fully supported)

\item {} 
\sphinxstyleliteralstrong{\sphinxupquote{dphi}} \textendash{} the size of the observer azimuthal viewing angle bin in degrees (only for 3D simulations, not fully
supported)

\item {} 
\sphinxstyleliteralstrong{\sphinxupquote{sim\_dims}} \textendash{} The number of dimensions of the hydro simulation used

\item {} 
\sphinxstyleliteralstrong{\sphinxupquote{h5}} \textendash{} specify if the format of the MCRaT output files is hdf5 files or not (denotes if using an old format or
a newer format of saving files)

\item {} 
\sphinxstyleliteralstrong{\sphinxupquote{photon\_type}} \textendash{} can be set to ‘s’, ‘i’, or left as None in order to select thermal synchrotron photons, injected photons, or all the photons in the simulation for analysis

\item {} 
\sphinxstyleliteralstrong{\sphinxupquote{energy\_range}} \textendash{} has units of keV, can be left as None to choose photons of all energy ranges for analysis or it can be set to an array with {[}min energy, max energy{]} e.g. {[}1, 10{]} for 1 to 10 keV (limits inclusive)

\end{itemize}

\item[{Returns}] \leavevmode
returns the time binned quantities of luminosity, luminosity error, number of photons in each bin, the
average eenrgy of photons in each bin, polarization, the stokes parameters, polarization error and
polarization angle

\end{description}\end{quote}

\end{fulllineitems}

\index{lcur\_var\_t() (in module mclib)@\spxentry{lcur\_var\_t()}\spxextra{in module mclib}}

\begin{fulllineitems}
\phantomsection\label{\detokenize{mclib:mclib.lcur_var_t}}\pysiglinewithargsret{\sphinxcode{\sphinxupquote{mclib.}}\sphinxbfcode{\sphinxupquote{lcur\_var\_t}}}{\emph{simid}, \emph{time\_start}, \emph{time\_end}, \emph{dt}, \emph{dt\_min=0.2}, \emph{liso\_c=1e+50}, \emph{units='erg/s'}, \emph{theta=1.0}, \emph{dtheta=1.0}, \emph{phi=0}, \emph{dphi=1}, \emph{sim\_dims=2}, \emph{h5=True}, \emph{photon\_type=None}, \emph{energy\_range=None}, \emph{use\_Lcrit=False}}{}
Produces time binned quantities for non-uniform time bins. The time bins must be larger than some critical
luminosity and some minimum dt that the user specifies.
\begin{quote}\begin{description}
\item[{Parameters}] \leavevmode\begin{itemize}
\item {} 
\sphinxstyleliteralstrong{\sphinxupquote{simid}} \textendash{} a string of the event file base name of the event file created in event\_h5 (everything but the .evt)

\item {} 
\sphinxstyleliteralstrong{\sphinxupquote{time\_start}} \textendash{} starting time of the light curve

\item {} 
\sphinxstyleliteralstrong{\sphinxupquote{time\_end}} \textendash{} end time of the light curve binning

\item {} 
\sphinxstyleliteralstrong{\sphinxupquote{dt}} \textendash{} initial dt of the time bins

\item {} 
\sphinxstyleliteralstrong{\sphinxupquote{dt\_min}} \textendash{} the minimum acceptable dt for the light curve

\item {} 
\sphinxstyleliteralstrong{\sphinxupquote{liso\_c}} \textendash{} the mimimum isotropic luminosity for a given time bin (in the same units specified by units)

\item {} 
\sphinxstyleliteralstrong{\sphinxupquote{units}} \textendash{} a string of the units of the light curve that will be produced (erg/s or cts/s)

\item {} 
\sphinxstyleliteralstrong{\sphinxupquote{theta}} \textendash{} the observer viewing angle in degrees

\item {} 
\sphinxstyleliteralstrong{\sphinxupquote{dtheta}} \textendash{} the size of the observer viewing angle bin in degrees (same as is specified in event\_h5 function)

\item {} 
\sphinxstyleliteralstrong{\sphinxupquote{phi}} \textendash{} azimuthal angle for mock observer in degrees (only for 3D simulations, not fully supported)

\item {} 
\sphinxstyleliteralstrong{\sphinxupquote{dphi}} \textendash{} the size of the observer azimuthal viewing angle bin in degrees (only for 3D simulations, not fully
supported)

\item {} 
\sphinxstyleliteralstrong{\sphinxupquote{sim\_dims}} \textendash{} The number of dimensions of the hydro simulation used

\item {} 
\sphinxstyleliteralstrong{\sphinxupquote{h5}} \textendash{} specify if the format of the MCRaT output files is hdf5 files or not (denotes if using an old format or
a newer format of saving files)

\item {} 
\sphinxstyleliteralstrong{\sphinxupquote{photon\_type}} \textendash{} can be set to ‘s’, ‘i’, or left as None in order to select thermal synchrotron photons, injected photons, or all the photons in the simulation for analysis

\item {} 
\sphinxstyleliteralstrong{\sphinxupquote{energy\_range}} \textendash{} has units of keV, can be left as None to choose photons of all energy ranges for analysis or it can be set to an array with {[}min energy, max energy{]} e.g. {[}1, 10{]} for 1 to 10 keV (limits inclusive)

\end{itemize}

\item[{Returns}] \leavevmode
returns the time binned quantities of luminosity, luminosity error, number of photons in each bin, the
average eenrgy of photons in each bin, polarization, the stokes parameters, polarization error and
polarization angle

\end{description}\end{quote}

\end{fulllineitems}

\index{lorentzBoostVectorized() (in module mclib)@\spxentry{lorentzBoostVectorized()}\spxextra{in module mclib}}

\begin{fulllineitems}
\phantomsection\label{\detokenize{mclib:mclib.lorentzBoostVectorized}}\pysiglinewithargsret{\sphinxcode{\sphinxupquote{mclib.}}\sphinxbfcode{\sphinxupquote{lorentzBoostVectorized}}}{\emph{boost}, \emph{P\_ph}}{}
Function to quickly lorentz boost a set of photon 4 momenta, and ensures that the 0 norm condition is met.
\begin{quote}\begin{description}
\item[{Parameters}] \leavevmode\begin{itemize}
\item {} 
\sphinxstyleliteralstrong{\sphinxupquote{boost}} \textendash{} The velocity vector of the frame that the photon will be boosted into. The shape of the array
should be (3,N), where N is the number of photons that will be lorentz boosted

\item {} 
\sphinxstyleliteralstrong{\sphinxupquote{P\_ph}} \textendash{} The photon’s 4 momentum that will be boosted into the desired frame of reference. The shape of the array
should be (4,N)

\end{itemize}

\item[{Returns}] \leavevmode
returns (4,N) array of the boosted photon’s 4 momenta

\end{description}\end{quote}

\end{fulllineitems}

\index{read\_event\_file() (in module mclib)@\spxentry{read\_event\_file()}\spxextra{in module mclib}}

\begin{fulllineitems}
\phantomsection\label{\detokenize{mclib:mclib.read_event_file}}\pysiglinewithargsret{\sphinxcode{\sphinxupquote{mclib.}}\sphinxbfcode{\sphinxupquote{read\_event\_file}}}{\emph{simid}, \emph{h5=True}}{}
An example function for reading the event files produced by event\_h5.
\begin{quote}\begin{description}
\item[{Parameters}] \leavevmode\begin{itemize}
\item {} 
\sphinxstyleliteralstrong{\sphinxupquote{simid}} \textendash{} the name of the event file to be read

\item {} 
\sphinxstyleliteralstrong{\sphinxupquote{h5}} \textendash{} switch to set whether the MCRaT simulation used hdf5 files (should always be set to true for the current versions of MCRaT)

\end{itemize}

\item[{Returns}] \leavevmode
returns the data from the event file

\end{description}\end{quote}

\end{fulllineitems}

\index{readanddecimate() (in module mclib)@\spxentry{readanddecimate()}\spxextra{in module mclib}}

\begin{fulllineitems}
\phantomsection\label{\detokenize{mclib:mclib.readanddecimate}}\pysiglinewithargsret{\sphinxcode{\sphinxupquote{mclib.}}\sphinxbfcode{\sphinxupquote{readanddecimate}}}{\emph{fnam}, \emph{inj\_radius}}{}
Legacy code from the python version of MCRaT to read and process FLASH files
\begin{quote}\begin{description}
\item[{Parameters}] \leavevmode\begin{itemize}
\item {} 
\sphinxstyleliteralstrong{\sphinxupquote{fnam}} \textendash{} string of directory and file name of the FLASH file that will be loaded and processed

\item {} 
\sphinxstyleliteralstrong{\sphinxupquote{inj\_radius}} \textendash{} The radius that photons are injected in cm

\end{itemize}

\item[{Returns}] \leavevmode
returns FLASH values of various quantities at cells

\end{description}\end{quote}

\end{fulllineitems}

\index{single\_cs() (in module mclib)@\spxentry{single\_cs()}\spxextra{in module mclib}}

\begin{fulllineitems}
\phantomsection\label{\detokenize{mclib:mclib.single_cs}}\pysiglinewithargsret{\sphinxcode{\sphinxupquote{mclib.}}\sphinxbfcode{\sphinxupquote{single\_cs}}}{\emph{P\_el}, \emph{P\_ph}}{}
This function conducts a single compton scatter in the electron rest frame. The photon has to be in the comoving
frame first. Legacy code from Python version of MCRaT.
\begin{quote}\begin{description}
\item[{Parameters}] \leavevmode\begin{itemize}
\item {} 
\sphinxstyleliteralstrong{\sphinxupquote{P\_el}} \textendash{} electron 4 momentum

\item {} 
\sphinxstyleliteralstrong{\sphinxupquote{P\_ph}} \textendash{} photon comoving 4 momentum

\end{itemize}

\item[{Returns}] \leavevmode
returns post-scattered photon 4 momentum in the comoving frame

\end{description}\end{quote}

\end{fulllineitems}

\index{single\_electron() (in module mclib)@\spxentry{single\_electron()}\spxextra{in module mclib}}

\begin{fulllineitems}
\phantomsection\label{\detokenize{mclib:mclib.single_electron}}\pysiglinewithargsret{\sphinxcode{\sphinxupquote{mclib.}}\sphinxbfcode{\sphinxupquote{single\_electron}}}{\emph{T}, \emph{P\_ph}}{}
This routine simulates one electron from a thermal distribution at a
temperature T. The electron has the correct angular distribution for a
photon traveling with direction P\_ph

\end{fulllineitems}

\index{spex() (in module mclib)@\spxentry{spex()}\spxextra{in module mclib}}

\begin{fulllineitems}
\phantomsection\label{\detokenize{mclib:mclib.spex}}\pysiglinewithargsret{\sphinxcode{\sphinxupquote{mclib.}}\sphinxbfcode{\sphinxupquote{spex}}}{\emph{simid}, \emph{numin}, \emph{numax}, \emph{tmin}, \emph{tmax}, \emph{units='erg/s'}, \emph{h5=True}, \emph{photon\_type=None}, \emph{calc\_pol=False}}{}
Produces spectra of phtons detected within any time interval
\begin{quote}\begin{description}
\item[{Parameters}] \leavevmode\begin{itemize}
\item {} 
\sphinxstyleliteralstrong{\sphinxupquote{simid}} \textendash{} a string of the event file base name of the event file created in event\_h5 (everything but the .evt)

\item {} 
\sphinxstyleliteralstrong{\sphinxupquote{numin}} \textendash{} array of energy values of the left most cutoff of the energy bins in the spectrum in keV

\item {} 
\sphinxstyleliteralstrong{\sphinxupquote{numax}} \textendash{} array of energy values of the right most cutoff of the energy bins in the spectrum in keV

\item {} 
\sphinxstyleliteralstrong{\sphinxupquote{tmin}} \textendash{} minimum of time bin that we are interested in collecting photons in to analyze

\item {} 
\sphinxstyleliteralstrong{\sphinxupquote{tmax}} \textendash{} max of time bin that we are interested in collecting photons in to analyze

\item {} 
\sphinxstyleliteralstrong{\sphinxupquote{units}} \textendash{} a string of the units of the spectrum that will be produced (erg/s or cts/s), each unit is then divided by the width of the enrgy bin in keV

\item {} 
\sphinxstyleliteralstrong{\sphinxupquote{h5}} \textendash{} specify if the format of the MCRaT output files is hdf5 files or not (denotes if using an old format or
a newer format of saving files)

\item {} 
\sphinxstyleliteralstrong{\sphinxupquote{photon\_type}} \textendash{} can be set to ‘s’, ‘i’, or left as None in order to select thermal synchrotron photons, injected photons, or all the photons in the simulation for analysis

\item {} 
\sphinxstyleliteralstrong{\sphinxupquote{energy\_range}} \textendash{} has units of keV, can be left as None to choose photons of all energy ranges for analysis or it can be set to an array with {[}min energy, max energy{]} e.g. {[}1, 10{]} for 1 to 10 keV (limits inclusive)

\end{itemize}

\item[{Returns}] \leavevmode
returns the spectrum with the specified units, for the photons in each energy bin

\end{description}\end{quote}

\end{fulllineitems}

\index{zero\_norm() (in module mclib)@\spxentry{zero\_norm()}\spxextra{in module mclib}}

\begin{fulllineitems}
\phantomsection\label{\detokenize{mclib:mclib.zero_norm}}\pysiglinewithargsret{\sphinxcode{\sphinxupquote{mclib.}}\sphinxbfcode{\sphinxupquote{zero\_norm}}}{\emph{P}}{}
Takes a photon 4 momentum and checks if it satisfies the 0 norm condition of photons. If not, it corrects the 4
mometum by assuming that the energy is correct.
\begin{quote}\begin{description}
\item[{Parameters}] \leavevmode
\sphinxstyleliteralstrong{\sphinxupquote{P}} \textendash{} photon 4 monetum

\item[{Returns}] \leavevmode
returns the correct 0 normed photon 4 momentum

\end{description}\end{quote}

\end{fulllineitems}



\chapter{read\_process\_files.py}
\label{\detokenize{read_process_files:module-read_process_files}}\label{\detokenize{read_process_files:read-process-files-py}}\label{\detokenize{read_process_files::doc}}\index{read\_process\_files (module)@\spxentry{read\_process\_files}\spxextra{module}}\index{Band() (in module read\_process\_files)@\spxentry{Band()}\spxextra{in module read\_process\_files}}

\begin{fulllineitems}
\phantomsection\label{\detokenize{read_process_files:read_process_files.Band}}\pysiglinewithargsret{\sphinxcode{\sphinxupquote{read\_process\_files.}}\sphinxbfcode{\sphinxupquote{Band}}}{\emph{x}, \emph{a}, \emph{b}, \emph{c}, \emph{d}}{}
Function that evaluates the Band function
\begin{quote}\begin{description}
\item[{Parameters}] \leavevmode\begin{itemize}
\item {} 
\sphinxstyleliteralstrong{\sphinxupquote{x}} \textendash{} x is the energy in keV

\item {} 
\sphinxstyleliteralstrong{\sphinxupquote{a}} \textendash{} alpha parameter

\item {} 
\sphinxstyleliteralstrong{\sphinxupquote{b}} \textendash{} beta parameter

\item {} 
\sphinxstyleliteralstrong{\sphinxupquote{c}} \textendash{} break energy in keV

\item {} 
\sphinxstyleliteralstrong{\sphinxupquote{d}} \textendash{} normalization of the returned spectrum

\end{itemize}

\item[{Returns}] \leavevmode
Returns the Band function evaluated for the parameters of interest (units of cts/s)

\end{description}\end{quote}

\end{fulllineitems}

\index{Goodman() (in module read\_process\_files)@\spxentry{Goodman()}\spxextra{in module read\_process\_files}}

\begin{fulllineitems}
\phantomsection\label{\detokenize{read_process_files:read_process_files.Goodman}}\pysiglinewithargsret{\sphinxcode{\sphinxupquote{read\_process\_files.}}\sphinxbfcode{\sphinxupquote{Goodman}}}{\emph{max\_spex}, \emph{max\_energy}}{}
Function that returns Goodman’s scalable spherical explosion spectra to compare against a spectra acquired by a
spherical explosion run in MCRAT. To compare this to simulation data, the simulation spectrum needs to be in units
of erg/s.
\begin{quote}\begin{description}
\item[{Parameters}] \leavevmode\begin{itemize}
\item {} 
\sphinxstyleliteralstrong{\sphinxupquote{max\_spex}} \textendash{} maximum value of the data’s spectrum.

\item {} 
\sphinxstyleliteralstrong{\sphinxupquote{max\_energy}} \textendash{} The energy bin where the maximum of the data lies

\end{itemize}

\item[{Returns}] \leavevmode
returns the scaled version of Goodman’s spectrum

\end{description}\end{quote}

\end{fulllineitems}

\index{amati() (in module read\_process\_files)@\spxentry{amati()}\spxextra{in module read\_process\_files}}

\begin{fulllineitems}
\phantomsection\label{\detokenize{read_process_files:read_process_files.amati}}\pysiglinewithargsret{\sphinxcode{\sphinxupquote{read\_process\_files.}}\sphinxbfcode{\sphinxupquote{amati}}}{\emph{value='o'}}{}
Return the Amati relationship and it’s 1 sigma dispersion as given by Tsutsui et al. (2009).
\begin{quote}\begin{description}
\item[{Parameters}] \leavevmode
\sphinxstyleliteralstrong{\sphinxupquote{value}} \textendash{} a string that can be ‘o’, ‘+’, or ‘-‘. The default is set to ‘o’ for the actual Amati relationship.
‘+’ gives the upper bound of uncertainty and ‘-‘ gives the lower bound of uncertainty.

\item[{Returns}] \leavevmode
returns arrays of the a and y values of the amati relation/ error in the relation

\end{description}\end{quote}

\end{fulllineitems}

\index{band\_hist\_data() (in module read\_process\_files)@\spxentry{band\_hist\_data()}\spxextra{in module read\_process\_files}}

\begin{fulllineitems}
\phantomsection\label{\detokenize{read_process_files:read_process_files.band_hist_data}}\pysiglinewithargsret{\sphinxcode{\sphinxupquote{read\_process\_files.}}\sphinxbfcode{\sphinxupquote{band\_hist\_data}}}{\emph{pickle\_file}, \emph{event\_files}, \emph{time\_start}, \emph{time\_end}, \emph{dt}, \emph{plotting=False}, \emph{save\_plot=False}, \emph{choose\_best=False}}{}
Function that collects a set of parameters from time resolved spectral fits of MCRaT data. This allows them to be
histogrammed to look at the distribution of alpha, beta and e\_pk for example.
\begin{quote}\begin{description}
\item[{Parameters}] \leavevmode\begin{itemize}
\item {} 
\sphinxstyleliteralstrong{\sphinxupquote{pickle\_file}} \textendash{} the name of the pickle file that will contain all of the spectral parameter values so it is only
necessary to run this lengthy calculation once.

\item {} 
\sphinxstyleliteralstrong{\sphinxupquote{event\_files}} \textendash{} The event files that will be analyzed to fit time resolved spectra and acquire the spectral
parameters, a 1D array of strings

\item {} 
\sphinxstyleliteralstrong{\sphinxupquote{time\_start}} \textendash{} The start time of the light curve that will be analyzed, a 1D array that is the same size as
event\_files

\item {} 
\sphinxstyleliteralstrong{\sphinxupquote{time\_end}} \textendash{} The end time of the light curve that will be analyzed, a 1D array that is the same size as
event\_files

\item {} 
\sphinxstyleliteralstrong{\sphinxupquote{dt}} \textendash{} The width of time bins that photons will be collected within to produce the time resolved spectra, this
will be across all of the event files that will be analyzed

\item {} 
\sphinxstyleliteralstrong{\sphinxupquote{plotting}} \textendash{} Switch to determine if the call to produce the light curves for each event file should plot
all the light curves that it produces so the user can inspect the spectral fits in each time interval

\item {} 
\sphinxstyleliteralstrong{\sphinxupquote{save\_plot}} \textendash{} Switch to determine if the call to produce the light curves for each event file should
save all the light curves that it produces (saves time in that the user doesn’t have to recall the
lcur\_param function to make plots of the light curves)

\item {} 
\sphinxstyleliteralstrong{\sphinxupquote{choose\_best}} \textendash{} Switch to ignore spectral parameters that haven’t been well constrained (they may not have
error bars)

\end{itemize}

\item[{Returns}] \leavevmode
Returns 2D arrays of alpha, beta, and break energies that contain all the time resolved parameter fits for
each event file that was passed to the function.

\end{description}\end{quote}

\end{fulllineitems}

\index{band\_hist\_plot() (in module read\_process\_files)@\spxentry{band\_hist\_plot()}\spxextra{in module read\_process\_files}}

\begin{fulllineitems}
\phantomsection\label{\detokenize{read_process_files:read_process_files.band_hist_plot}}\pysiglinewithargsret{\sphinxcode{\sphinxupquote{read\_process\_files.}}\sphinxbfcode{\sphinxupquote{band\_hist\_plot}}}{\emph{pickle\_file}, \emph{plotting=True}, \emph{save\_plot=False}, \emph{scale=1}}{}
Function to plot the histogrammed Band/COMP spectral parameters.
\begin{quote}\begin{description}
\item[{Parameters}] \leavevmode\begin{itemize}
\item {} 
\sphinxstyleliteralstrong{\sphinxupquote{pickle\_file}} \textendash{} The produced pickle file name from the band\_hist\_data function

\item {} 
\sphinxstyleliteralstrong{\sphinxupquote{plotting}} \textendash{} Switch to show the produced plots

\item {} 
\sphinxstyleliteralstrong{\sphinxupquote{save\_plot}} \textendash{} Switch to save the plotted histograms

\item {} 
\sphinxstyleliteralstrong{\sphinxupquote{scale}} \textendash{} Scaling to scale the histograms if necessary

\end{itemize}

\item[{Returns}] \leavevmode
Returns 2D arrays of alpha, beta, and break energies that contain all the time resolved parameter fits for
each event file that was passed to the band\_hist\_data function.

\end{description}\end{quote}

\end{fulllineitems}

\index{blackbody() (in module read\_process\_files)@\spxentry{blackbody()}\spxextra{in module read\_process\_files}}

\begin{fulllineitems}
\phantomsection\label{\detokenize{read_process_files:read_process_files.blackbody}}\pysiglinewithargsret{\sphinxcode{\sphinxupquote{read\_process\_files.}}\sphinxbfcode{\sphinxupquote{blackbody}}}{\emph{x}, \emph{t}, \emph{d}}{}
Function that produces a blackbody spectrum.
\begin{quote}\begin{description}
\item[{Parameters}] \leavevmode\begin{itemize}
\item {} 
\sphinxstyleliteralstrong{\sphinxupquote{x}} \textendash{} Energy in keV

\item {} 
\sphinxstyleliteralstrong{\sphinxupquote{t}} \textendash{} temp in kelvin

\item {} 
\sphinxstyleliteralstrong{\sphinxupquote{d}} \textendash{} normalization of the spectrum

\end{itemize}

\item[{Returns}] \leavevmode
returns the blackbody spectrum parameterized by some temperature at the energies of interest

\end{description}\end{quote}

\end{fulllineitems}

\index{cfit() (in module read\_process\_files)@\spxentry{cfit()}\spxextra{in module read\_process\_files}}

\begin{fulllineitems}
\phantomsection\label{\detokenize{read_process_files:read_process_files.cfit}}\pysiglinewithargsret{\sphinxcode{\sphinxupquote{read\_process\_files.}}\sphinxbfcode{\sphinxupquote{cfit}}}{\emph{event\_file}, \emph{time\_start}, \emph{time\_end}, \emph{dnulog=0.1}, \emph{hdf5=True}, \emph{plotting=False}, \emph{save\_plot=False}, \emph{save\_spex=None}, \emph{append=False}, \emph{photon\_type=None}, \emph{calc\_pol=False}, \emph{calc\_synch=False}}{}
Function that fits either the COMP function of the Band function to time integrated or time resolved spectra from
MCRaT and also acquires the 1 sigma error in the fitted values. The best fit spectra, accounting for the extra
degree of freedom is chosen by conducting an F-test.
\begin{quote}\begin{description}
\item[{Parameters}] \leavevmode\begin{itemize}
\item {} 
\sphinxstyleliteralstrong{\sphinxupquote{event\_file}} \textendash{} the base name of the event file that holds all of the info about the mock observed photons
from MCRaT

\item {} 
\sphinxstyleliteralstrong{\sphinxupquote{time\_start}} \textendash{} the start time of a given time bin to collect photons into a spectrum

\item {} 
\sphinxstyleliteralstrong{\sphinxupquote{time\_end}} \textendash{} the end of the time bin for collecting photons into a spectrum to be fit by this function

\item {} 
\sphinxstyleliteralstrong{\sphinxupquote{dnulog}} \textendash{} the spacing between energy bins when binning the mock observed MCRaT photons (in keV)

\item {} 
\sphinxstyleliteralstrong{\sphinxupquote{hdf5}} \textendash{} switch to let the function know if the MCRaT output files are hdf5 files or not

\item {} 
\sphinxstyleliteralstrong{\sphinxupquote{plotting}} \textendash{} switch to let the function knwo that it should plot the MCraT spectra and the fits to the spectra

\item {} 
\sphinxstyleliteralstrong{\sphinxupquote{save\_plot}} \textendash{} switch to let the function know if it should save the plots

\item {} 
\sphinxstyleliteralstrong{\sphinxupquote{save\_spex}} \textendash{} string to let the function save the tabulated spectrum to a file given by this string

\item {} 
\sphinxstyleliteralstrong{\sphinxupquote{append}} \textendash{} switch to let the function know if it should append data to the file given by save\_spex

\item {} 
\sphinxstyleliteralstrong{\sphinxupquote{photon\_type}} \textendash{} can be set to ‘s’, ‘i’, or left as None in order to select thermal synchrotron photons, injected photons, or all the photons in the simulation for analysis

\item {} 
\sphinxstyleliteralstrong{\sphinxupquote{calc\_pol}} \textendash{} switch to determine of polarization as a function of energy should be calculated and plotted

\end{itemize}

\item[{Returns}] \leavevmode
returns the best fitted parameters to the function, a (3,2) array that has each parameters’ (alpha, beta
and break energy) upper and lower 1 sigma errors, and a character denoting the function that provided the best fit (‘c’ for
COMP or ‘b’ for Band)

\end{description}\end{quote}

\end{fulllineitems}

\index{comp() (in module read\_process\_files)@\spxentry{comp()}\spxextra{in module read\_process\_files}}

\begin{fulllineitems}
\phantomsection\label{\detokenize{read_process_files:read_process_files.comp}}\pysiglinewithargsret{\sphinxcode{\sphinxupquote{read\_process\_files.}}\sphinxbfcode{\sphinxupquote{comp}}}{\emph{x}, \emph{a}, \emph{c}, \emph{d}}{}
returns the Comptonized (COMP) spectrum for a given set of energies
\begin{quote}\begin{description}
\item[{Parameters}] \leavevmode\begin{itemize}
\item {} 
\sphinxstyleliteralstrong{\sphinxupquote{x}} \textendash{} the energies that the Comptonized function will be evaluated at (should be in keV)

\item {} 
\sphinxstyleliteralstrong{\sphinxupquote{a}} \textendash{} the alpha parameter

\item {} 
\sphinxstyleliteralstrong{\sphinxupquote{c}} \textendash{} the break energy

\item {} 
\sphinxstyleliteralstrong{\sphinxupquote{d}} \textendash{} the normalization of the spectrum

\end{itemize}

\item[{Returns}] \leavevmode
returns the COMP spectrum at the energies of interest (units of cts/s)

\end{description}\end{quote}

\end{fulllineitems}

\index{fluidGammaVsTheta() (in module read\_process\_files)@\spxentry{fluidGammaVsTheta()}\spxextra{in module read\_process\_files}}

\begin{fulllineitems}
\phantomsection\label{\detokenize{read_process_files:read_process_files.fluidGammaVsTheta}}\pysiglinewithargsret{\sphinxcode{\sphinxupquote{read\_process\_files.}}\sphinxbfcode{\sphinxupquote{fluidGammaVsTheta}}}{\emph{fluid\_dir}, \emph{r\_interest}, \emph{t\_interest}, \emph{fps}, \emph{theta\_max}}{}
Function to calculate the lorentz factor profile as a function of theta directly from the hydrodynamic files.
It calculates which frame to analyze based off of what radius the user is interested in and the times in the outflow
the user is interested in
\begin{quote}\begin{description}
\item[{Parameters}] \leavevmode\begin{itemize}
\item {} 
\sphinxstyleliteralstrong{\sphinxupquote{fluid\_dir}} \textendash{} String that points to the directory of the hydrodynamic files, should include the base name of the
hydro file (excluding any numbers)

\item {} 
\sphinxstyleliteralstrong{\sphinxupquote{r\_interest}} \textendash{} The radius at which the user wants the flow to be analyzed

\item {} 
\sphinxstyleliteralstrong{\sphinxupquote{t\_interest}} \textendash{} An array of times in the ouflow that the user wants to be analyzed

\item {} 
\sphinxstyleliteralstrong{\sphinxupquote{fps}} \textendash{} The frames per second of the passed hydrodynamic simulation

\item {} 
\sphinxstyleliteralstrong{\sphinxupquote{theta\_max}} \textendash{} The maximum angle to get the lorentz factor profile

\end{itemize}

\item[{Returns}] \leavevmode
All the theta data points, in degrees, the average lorentz profile at the angle data points, and the array
of the times that the user was interested in

\end{description}\end{quote}

\end{fulllineitems}

\index{get\_FERMI\_best\_data() (in module read\_process\_files)@\spxentry{get\_FERMI\_best\_data()}\spxextra{in module read\_process\_files}}

\begin{fulllineitems}
\phantomsection\label{\detokenize{read_process_files:read_process_files.get_FERMI_best_data}}\pysiglinewithargsret{\sphinxcode{\sphinxupquote{read\_process\_files.}}\sphinxbfcode{\sphinxupquote{get\_FERMI\_best\_data}}}{}{}
A function to acquire data about the FERMI Best GRB sample, as is saved in the file named FERMI\_BEST\_GRB.dat.
The data is from Yu et al. (2016).
\begin{quote}\begin{description}
\item[{Returns}] \leavevmode
returns arrays of the Band or COMP function fitted GRB spectral parameters

\end{description}\end{quote}

\end{fulllineitems}

\index{get\_amati\_rel() (in module read\_process\_files)@\spxentry{get\_amati\_rel()}\spxextra{in module read\_process\_files}}

\begin{fulllineitems}
\phantomsection\label{\detokenize{read_process_files:read_process_files.get_amati_rel}}\pysiglinewithargsret{\sphinxcode{\sphinxupquote{read\_process\_files.}}\sphinxbfcode{\sphinxupquote{get\_amati\_rel}}}{\emph{simid\_array}, \emph{time\_start}, \emph{time\_end}, \emph{save\_plot=False}, \emph{h5=False}}{}
A function to plot the MCRaT analyzed GRB alongside the amati relationship and a list of observed GRBs as
provided by GRB\_list.dat (observed GRBS are from Nava et al. (2012))
\begin{quote}\begin{description}
\item[{Parameters}] \leavevmode\begin{itemize}
\item {} 
\sphinxstyleliteralstrong{\sphinxupquote{simid\_array}} \textendash{} An array of the event file name bases of the MCRaT result that will be plotted

\item {} 
\sphinxstyleliteralstrong{\sphinxupquote{t\_start}} \textendash{} An array of the start times for each event file passed to simid\_array

\item {} 
\sphinxstyleliteralstrong{\sphinxupquote{t\_end}} \textendash{} An array of the end times for each event file passed to simid\_array

\item {} 
\sphinxstyleliteralstrong{\sphinxupquote{save\_plot}} \textendash{} Switch to save the plot as a PDF or not

\item {} 
\sphinxstyleliteralstrong{\sphinxupquote{h5}} \textendash{} Switch to specify if the MCRaT output files were HDF5 format or not, default is True. Should be set to
False when an old version of MCRaT was used which outputtted the data in text files

\end{itemize}

\item[{Returns}] \leavevmode
No returns

\end{description}\end{quote}

\end{fulllineitems}

\index{get\_lundman\_pol\_data() (in module read\_process\_files)@\spxentry{get\_lundman\_pol\_data()}\spxextra{in module read\_process\_files}}

\begin{fulllineitems}
\phantomsection\label{\detokenize{read_process_files:read_process_files.get_lundman_pol_data}}\pysiglinewithargsret{\sphinxcode{\sphinxupquote{read\_process\_files.}}\sphinxbfcode{\sphinxupquote{get\_lundman\_pol\_data}}}{\emph{p=4}, \emph{thetaj=0.01}}{}
This function gets the appropriate values of polarization vs theta for various values of p and theta\_j from the
plots of Lundman et al. (2014b). The only accepted parameter combinations are p=4 with thetaj=0.01 or 0.1 and
p=2 with thetaj=0.01
\begin{quote}\begin{description}
\item[{Parameters}] \leavevmode\begin{itemize}
\item {} 
\sphinxstyleliteralstrong{\sphinxupquote{p}} \textendash{} The p value for the width of the analytic jet from Lundman et al. (2014b)

\item {} 
\sphinxstyleliteralstrong{\sphinxupquote{thetaj}} \textendash{} The opening angle of the analytic jet profile from Lundman et al. (2014b)

\end{itemize}

\item[{Returns}] \leavevmode
Returns a (nx2) array where n is the number of angles points read in fro the file and the columns are the
theta/thetaj values followed by the polarization degree

\end{description}\end{quote}

\end{fulllineitems}

\index{get\_yonetoku\_rel() (in module read\_process\_files)@\spxentry{get\_yonetoku\_rel()}\spxextra{in module read\_process\_files}}

\begin{fulllineitems}
\phantomsection\label{\detokenize{read_process_files:read_process_files.get_yonetoku_rel}}\pysiglinewithargsret{\sphinxcode{\sphinxupquote{read\_process\_files.}}\sphinxbfcode{\sphinxupquote{get\_yonetoku\_rel}}}{\emph{simid\_array}, \emph{t\_start}, \emph{t\_end}, \emph{dt=1}, \emph{h5=False}, \emph{save\_plot=False}}{}
A function to plot the MCRaT analyzed GRB alongside the yonetoku relationship and a list of observed GRBs as
provided by GRB\_list.dat (observed GRBS are from Nava et al. (2012))
\begin{quote}\begin{description}
\item[{Parameters}] \leavevmode\begin{itemize}
\item {} 
\sphinxstyleliteralstrong{\sphinxupquote{simid\_array}} \textendash{} An array of the event file name bases of the MCRaT result that will be plotted

\item {} 
\sphinxstyleliteralstrong{\sphinxupquote{t\_start}} \textendash{} An array of the start times for each event file passed to simid\_array

\item {} 
\sphinxstyleliteralstrong{\sphinxupquote{t\_end}} \textendash{} An array of the end times for each event file passed to simid\_array

\item {} 
\sphinxstyleliteralstrong{\sphinxupquote{dt}} \textendash{} The time bin width, default is 1 following how the relationship is derived

\item {} 
\sphinxstyleliteralstrong{\sphinxupquote{h5}} \textendash{} Switch to specify if the MCRaT output files were HDF5 format or not, default is True. Should be set to
False when an old version of MCRaT was used which outputtted the data in text files

\item {} 
\sphinxstyleliteralstrong{\sphinxupquote{save\_plot}} \textendash{} Switch to save the plot as a PDF or not

\end{itemize}

\item[{Returns}] \leavevmode
Returns arrays of the plotted peak energy, its error, the luminosity and the erro in the luminosity for
each event file that was provided to simid\_array

\end{description}\end{quote}

\end{fulllineitems}

\index{golenetskii() (in module read\_process\_files)@\spxentry{golenetskii()}\spxextra{in module read\_process\_files}}

\begin{fulllineitems}
\phantomsection\label{\detokenize{read_process_files:read_process_files.golenetskii}}\pysiglinewithargsret{\sphinxcode{\sphinxupquote{read\_process\_files.}}\sphinxbfcode{\sphinxupquote{golenetskii}}}{\emph{value='o'}}{}
Return the golenetskii relationship and it’s 2 sigma dispersion as given by Lu et al. (2012).
\begin{quote}\begin{description}
\item[{Parameters}] \leavevmode
\sphinxstyleliteralstrong{\sphinxupquote{value}} \textendash{} a string that can be ‘o’, ‘+’, or ‘-‘. The default is set to ‘o’ for the actual golenetskii relationship.
‘+’ gives the upper bound of uncertainty and ‘-‘ gives the lower bound of uncertainty.

\item[{Returns}] \leavevmode
returns arrays of the a and y values of the relation/ error in the relation

\end{description}\end{quote}

\end{fulllineitems}

\index{lcur\_param() (in module read\_process\_files)@\spxentry{lcur\_param()}\spxextra{in module read\_process\_files}}

\begin{fulllineitems}
\phantomsection\label{\detokenize{read_process_files:read_process_files.lcur_param}}\pysiglinewithargsret{\sphinxcode{\sphinxupquote{read\_process\_files.}}\sphinxbfcode{\sphinxupquote{lcur\_param}}}{\emph{event\_file}, \emph{time\_start}, \emph{time\_end}, \emph{dt=1}, \emph{phi\_angle=0}, \emph{delta\_theta=1}, \emph{unit='erg/s'}, \emph{plotting=True}, \emph{save\_plot=False}, \emph{choose\_best=False}, \emph{lc\_only=False}, \emph{pol\_only=False}, \emph{dim=2}, \emph{hdf5=True}, \emph{save\_spex=None}, \emph{photon\_type=None}, \emph{energy\_range=None}}{}
This function takes the MCRaT ‘detected’ photons from the event file and time bins them to produce light curves. The
photons in a given time bin are then fitted with either the COMP or Band functions.
\begin{quote}\begin{description}
\item[{Parameters}] \leavevmode\begin{itemize}
\item {} 
\sphinxstyleliteralstrong{\sphinxupquote{event\_file}} \textendash{} the base name of the event file that holds all of the info about the mock observed photons
from MCRaT

\item {} 
\sphinxstyleliteralstrong{\sphinxupquote{time\_start}} \textendash{} the start time to begin binning the photons

\item {} 
\sphinxstyleliteralstrong{\sphinxupquote{time\_end}} \textendash{} the end time to stop binning the photons

\item {} 
\sphinxstyleliteralstrong{\sphinxupquote{dt}} \textendash{} the light curve time bin width

\item {} 
\sphinxstyleliteralstrong{\sphinxupquote{phi\_angle}} \textendash{} unused parameter, meant for future 3D development

\item {} 
\sphinxstyleliteralstrong{\sphinxupquote{delta\_theta}} \textendash{} delta theta of the observer viewing angle for accepting photons, in degrees. Should be the same
value as what is passed into event\_h5’s dtheta\_deg

\item {} 
\sphinxstyleliteralstrong{\sphinxupquote{unit}} \textendash{} A string of the units that will be used for the light curve/spectrum ‘erg/s’ or ‘cts/s’

\item {} 
\sphinxstyleliteralstrong{\sphinxupquote{plotting}} \textendash{} Switch to allow the function to show the plots of the light curve/spectral parameters

\item {} 
\sphinxstyleliteralstrong{\sphinxupquote{save\_plot}} \textendash{} Switch to allow the function to save the produced plot

\item {} 
\sphinxstyleliteralstrong{\sphinxupquote{choose\_best}} \textendash{} Switch that permits the function to overwrite spectral parameters that do not have error bars
to be ignored if set to True. If this is set to False all spectral parameters will be considered/plotted

\item {} 
\sphinxstyleliteralstrong{\sphinxupquote{lc\_only}} \textendash{} switch that tells the function to only plot the light curve and peak energy of the time resolved
spectra and none of the other parametres

\item {} 
\sphinxstyleliteralstrong{\sphinxupquote{pol\_only}} \textendash{} Switch to allow the function to only plot the light curve and the polarization when set to True

\item {} 
\sphinxstyleliteralstrong{\sphinxupquote{dim}} \textendash{} The number of dimensions of the MCRaT simulation that is being analyzed. There is no support for 3D at
this time so this switch should be set to 2, which is it’s default value

\item {} 
\sphinxstyleliteralstrong{\sphinxupquote{hdf5}} \textendash{} Switch to specify if the MCRaT output files were HDF5 format or not, default is True. Should be set to
False when an old version of MCRaT was used which outputtted the data in text files

\item {} 
\sphinxstyleliteralstrong{\sphinxupquote{save\_spex}} \textendash{} Switch that allows the time resolved spectra to be saved to a text file

\item {} 
\sphinxstyleliteralstrong{\sphinxupquote{photon\_type}} \textendash{} can be set to ‘s’, ‘i’, or left as None in order to select thermal synchrotron photons, injected photons, or all the photons in the simulation for analysis

\item {} 
\sphinxstyleliteralstrong{\sphinxupquote{energy\_range}} \textendash{} has units of keV, can be left as None to choose photons of all energy ranges for analysis or it can be set to an array with {[}min energy, max energy{]} e.g. {[}1, 10{]} for 1 to 10 keV (limits inclusive)

\end{itemize}

\item[{Returns}] \leavevmode
returns arrays of the various time resolved parameters and their errors. The order of the variables
returned are: light curve, light curve error, alpha,beta,break energy, errors in the spectral parameters
(a (times.size,3,2) array where times is the start of the light curve time bins), start times of the time bins, the best fit model in each time bin,
polarization, stokes I, stokes Q, stokes U, stokes V, polarization and polarization angle errors (a ((times.size,2)) array), polarization angle,
number of photons in each time bin

\end{description}\end{quote}

\end{fulllineitems}

\index{lcur\_param\_plot() (in module read\_process\_files)@\spxentry{lcur\_param\_plot()}\spxextra{in module read\_process\_files}}

\begin{fulllineitems}
\phantomsection\label{\detokenize{read_process_files:read_process_files.lcur_param_plot}}\pysiglinewithargsret{\sphinxcode{\sphinxupquote{read\_process\_files.}}\sphinxbfcode{\sphinxupquote{lcur\_param\_plot}}}{\emph{event\_file}, \emph{lcur}, \emph{lcur\_e}, \emph{alpha}, \emph{beta}, \emph{e\_o}, \emph{err}, \emph{t}, \emph{model}, \emph{P}, \emph{I}, \emph{Q}, \emph{U}, \emph{V}, \emph{Perr}, \emph{P\_angle}, \emph{num\_scatt}, \emph{dt=1}, \emph{lc\_only=False}, \emph{pol\_only=False}, \emph{plotting=True}, \emph{h5=False}, \emph{liso\_c=None}, \emph{plot\_optical=True}}{}
The plotting function for the results of lcur\_param and lcur\_param\_var\_t functions.
\begin{quote}\begin{description}
\item[{Parameters}] \leavevmode\begin{itemize}
\item {} 
\sphinxstyleliteralstrong{\sphinxupquote{event\_file}} \textendash{} The event file name, a string that will be ncorporated into the saved pdf of the plot that is produced

\item {} 
\sphinxstyleliteralstrong{\sphinxupquote{lcur}} \textendash{} the values of the light curve at each time bin as an array

\item {} 
\sphinxstyleliteralstrong{\sphinxupquote{lcur\_e}} \textendash{} the values of the error in he light curve values at each time bin

\item {} 
\sphinxstyleliteralstrong{\sphinxupquote{alpha}} \textendash{} an array of the fitted alpha parameters for each time resolved spectrum

\item {} 
\sphinxstyleliteralstrong{\sphinxupquote{beta}} \textendash{} an array of the fitted beta parameters for each time resolved spectrum

\item {} 
\sphinxstyleliteralstrong{\sphinxupquote{e\_o}} \textendash{} an array of the fitted break energies for each time resolved spectrum

\item {} 
\sphinxstyleliteralstrong{\sphinxupquote{err}} \textendash{} A (t.size,3,2) array of errors in the spectral parameters

\item {} 
\sphinxstyleliteralstrong{\sphinxupquote{t}} \textendash{} An array of time bins both beginning and end that will be plotted

\item {} 
\sphinxstyleliteralstrong{\sphinxupquote{model}} \textendash{} An array of model types for the types of spectrum that best fit each time binned spectrum

\item {} 
\sphinxstyleliteralstrong{\sphinxupquote{P}} \textendash{} An array of he polarization at each time bin

\item {} 
\sphinxstyleliteralstrong{\sphinxupquote{I}} \textendash{} 

\item {} 
\sphinxstyleliteralstrong{\sphinxupquote{Q}} \textendash{} 

\item {} 
\sphinxstyleliteralstrong{\sphinxupquote{U}} \textendash{} 

\item {} 
\sphinxstyleliteralstrong{\sphinxupquote{V}} \textendash{} 

\item {} 
\sphinxstyleliteralstrong{\sphinxupquote{Perr}} \textendash{} The error in the polarization at each time bin

\item {} 
\sphinxstyleliteralstrong{\sphinxupquote{P\_angle}} \textendash{} The polarization angle of the photons detected in each time bin

\item {} 
\sphinxstyleliteralstrong{\sphinxupquote{dt}} \textendash{} the size of the time bin if the time bins are uniformly size otherwise set it to be a negative umber

\item {} 
\sphinxstyleliteralstrong{\sphinxupquote{lc\_only}} \textendash{} Switch to denote if the function should only plot the light curve or not

\item {} 
\sphinxstyleliteralstrong{\sphinxupquote{pol\_only}} \textendash{} Switch to denote if the function should plot just the light curve and the polarization parameters

\item {} 
\sphinxstyleliteralstrong{\sphinxupquote{plotting}} \textendash{} Switch, set to True by default, to show the plot produced

\item {} 
\sphinxstyleliteralstrong{\sphinxupquote{h5}} \textendash{} A switch to denote that the MCRaT simulation that is being analyzed produced HDF5 files (for backwatds compatibility)

\item {} 
\sphinxstyleliteralstrong{\sphinxupquote{liso\_c}} \textendash{} The value of the critical isotropic luminosity that was used to create the variable time bins

\end{itemize}

\item[{Returns}] \leavevmode
No returns

\end{description}\end{quote}

\end{fulllineitems}

\index{lcur\_param\_var\_t() (in module read\_process\_files)@\spxentry{lcur\_param\_var\_t()}\spxextra{in module read\_process\_files}}

\begin{fulllineitems}
\phantomsection\label{\detokenize{read_process_files:read_process_files.lcur_param_var_t}}\pysiglinewithargsret{\sphinxcode{\sphinxupquote{read\_process\_files.}}\sphinxbfcode{\sphinxupquote{lcur\_param\_var\_t}}}{\emph{event\_file}, \emph{time\_start}, \emph{time\_end}, \emph{dt=0.2}, \emph{dt\_min=0.2}, \emph{phi\_angle=0}, \emph{delta\_theta=1}, \emph{liso\_c=None}, \emph{unit='erg/s'}, \emph{plotting=True}, \emph{save\_plot=False}, \emph{choose\_best=False}, \emph{lc\_only=False}, \emph{pol\_only=False}, \emph{dim=2}, \emph{riken\_switch=False}, \emph{hdf5=True}, \emph{save\_spex=None}, \emph{photon\_type=None}, \emph{energy\_range=None}, \emph{use\_Lcrit=False}, \emph{plot3=False}}{}
This function takes the MCRaT ‘detected’ photons from the event file and time bins them to produce light curves with
variable time bins. The photons in a given time bin are then fitted with either the COMP or Band functions.

This function should be called after viewing the data in uniform time bins. It uses a minimum luminosity cutoff to
form time bins, if this critical limit is achieved, the time bin is recorded otherwise the function will keep
increasing a giving time bin in a light curve to ensure that the critical luminosity is achieved.
\begin{quote}\begin{description}
\item[{Parameters}] \leavevmode\begin{itemize}
\item {} 
\sphinxstyleliteralstrong{\sphinxupquote{event\_file}} \textendash{} the base name of the event file that holds all of the info about the mock observed photons
from MCRaT

\item {} 
\sphinxstyleliteralstrong{\sphinxupquote{time}} \textendash{} a python list or numpy array with the start time and the end time in chronological order e.g. time={[}0, 35{]}

\item {} 
\sphinxstyleliteralstrong{\sphinxupquote{dt}} \textendash{} the light curve time bin width

\item {} 
\sphinxstyleliteralstrong{\sphinxupquote{phi\_angle}} \textendash{} unused parameter, meant for future 3D development

\item {} 
\sphinxstyleliteralstrong{\sphinxupquote{delta\_theta}} \textendash{} delta theta of the observer viewing angle for accepting photons, in degrees. Should be the same
value as what is passed into event\_h5’s dtheta\_deg

\item {} 
\sphinxstyleliteralstrong{\sphinxupquote{liso\_c}} \textendash{} The critical luminosity that has to be achieved in order for the function to close a given time bin

\item {} 
\sphinxstyleliteralstrong{\sphinxupquote{unit}} \textendash{} A string of the units that will be used for the light curve/spectrum ‘erg/s’ or ‘cts/s’

\item {} 
\sphinxstyleliteralstrong{\sphinxupquote{plotting}} \textendash{} Switch to allow the function to show the plots of the light curve/spectral parameters

\item {} 
\sphinxstyleliteralstrong{\sphinxupquote{save\_plot}} \textendash{} Switch to allow the function to save the produced plot

\item {} 
\sphinxstyleliteralstrong{\sphinxupquote{choose\_best}} \textendash{} Switch that permits the function to overwrite spectral parameters that do not have error bars
to be ignored if set to True. If this is set to False all spectral parameters will be considered/plotted

\item {} 
\sphinxstyleliteralstrong{\sphinxupquote{lc\_only}} \textendash{} switch that tells the function to only plot the light curve and peak energy of the time resolved
spectra and none of the other parametres

\item {} 
\sphinxstyleliteralstrong{\sphinxupquote{pol\_only}} \textendash{} Switch to allow the function to only plot the light curve and the polarization when set to True

\item {} 
\sphinxstyleliteralstrong{\sphinxupquote{dim}} \textendash{} The number of dimensions of the MCRaT simulation that is being analyzed. There is no support for 3D at
this time so this switch should be set to 2, which is it’s default value

\item {} 
\sphinxstyleliteralstrong{\sphinxupquote{hdf5}} \textendash{} Switch to specify if the MCRaT output files were HDF5 format or not, default is True. Should be set to
False when an old version of MCRaT was used which outputted the data in text files

\item {} 
\sphinxstyleliteralstrong{\sphinxupquote{save\_spex}} \textendash{} Switch that allows the time resolved spectra to be saved to a text file

\item {} 
\sphinxstyleliteralstrong{\sphinxupquote{photon\_type}} \textendash{} can be set to ‘s’, ‘i’, or left as None in order to select thermal synchrotron photons, injected photons, or all the photons in the simulation for analysis

\item {} 
\sphinxstyleliteralstrong{\sphinxupquote{energy\_range}} \textendash{} has units of keV, can be left as None to choose photons of all energy ranges for analysis or it can be set to an array with {[}min energy, max energy{]} e.g. {[}1, 10{]} for 1 to 10 keV (limits inclusive)

\end{itemize}

\item[{Returns}] \leavevmode
returns arrays of the various time resolved parameters and their errors. The order of the variables
returned are: light curve, light curve error, alpha,beta,break energy, errors in the spectral parameters
(a (times.size,3,2) array where times is the start of the light curve time bins), start times of the time bins, the best fit model in each time bin,
polarization, stokes I, stokes Q, stokes U, stokes V, polarization and polarization angle errors (a ((times.size,2)) array), polarization angle,
number of photons in each time bin

\end{description}\end{quote}

\end{fulllineitems}

\index{lundmanPolarizationGamma() (in module read\_process\_files)@\spxentry{lundmanPolarizationGamma()}\spxextra{in module read\_process\_files}}

\begin{fulllineitems}
\phantomsection\label{\detokenize{read_process_files:read_process_files.lundmanPolarizationGamma}}\pysiglinewithargsret{\sphinxcode{\sphinxupquote{read\_process\_files.}}\sphinxbfcode{\sphinxupquote{lundmanPolarizationGamma}}}{\emph{p}, \emph{theta\_j}, \emph{gamma\_0}, \emph{max\_angle}}{}
Function that calculates Lundman et al.’s (2014b) lorentz factor profile as a function of theta
\begin{quote}\begin{description}
\item[{Parameters}] \leavevmode\begin{itemize}
\item {} 
\sphinxstyleliteralstrong{\sphinxupquote{p}} \textendash{} The p value for the width of the analytic jet from Lundman et al. (2014b)

\item {} 
\sphinxstyleliteralstrong{\sphinxupquote{theta\_j}} \textendash{} The opening angle of the analytic jet profile from Lundman et al. (2014b)

\item {} 
\sphinxstyleliteralstrong{\sphinxupquote{gamma\_0}} \textendash{} The jet core lorentz factor

\item {} 
\sphinxstyleliteralstrong{\sphinxupquote{max\_angle}} \textendash{} The maximum angle to calculate the lorentz factor of

\end{itemize}

\item[{Returns}] \leavevmode
Returns arrays of the angles and the lorentz factors at those angles

\end{description}\end{quote}

\end{fulllineitems}

\index{param\_err() (in module read\_process\_files)@\spxentry{param\_err()}\spxextra{in module read\_process\_files}}

\begin{fulllineitems}
\phantomsection\label{\detokenize{read_process_files:read_process_files.param_err}}\pysiglinewithargsret{\sphinxcode{\sphinxupquote{read\_process\_files.}}\sphinxbfcode{\sphinxupquote{param\_err}}}{\emph{x}, \emph{data}, \emph{err}, \emph{best\_par}, \emph{chi\_sq}, \emph{par}}{}
Function to calculate the 1 sigma errors in the comptonized fit to the spectrum.
\begin{quote}\begin{description}
\item[{Parameters}] \leavevmode\begin{itemize}
\item {} 
\sphinxstyleliteralstrong{\sphinxupquote{x}} \textendash{} energies of the spectrum in keV

\item {} 
\sphinxstyleliteralstrong{\sphinxupquote{data}} \textendash{} The spectrum data points that are being fitted

\item {} 
\sphinxstyleliteralstrong{\sphinxupquote{err}} \textendash{} the errors in each data point

\item {} 
\sphinxstyleliteralstrong{\sphinxupquote{best\_par}} \textendash{} the best fit parameters of the Band function to the MCRaT spectrum

\item {} 
\sphinxstyleliteralstrong{\sphinxupquote{chi\_sq}} \textendash{} The value of chi squared of the best fit (value of best\_par)

\item {} 
\sphinxstyleliteralstrong{\sphinxupquote{par}} \textendash{} The Band spectrum parameter of interest that we would like to find the errors of

\end{itemize}

\item[{Returns}] \leavevmode
returns a (1x2) array of the negative error bar and the positive error bar

\end{description}\end{quote}

\end{fulllineitems}

\index{param\_err\_comp() (in module read\_process\_files)@\spxentry{param\_err\_comp()}\spxextra{in module read\_process\_files}}

\begin{fulllineitems}
\phantomsection\label{\detokenize{read_process_files:read_process_files.param_err_comp}}\pysiglinewithargsret{\sphinxcode{\sphinxupquote{read\_process\_files.}}\sphinxbfcode{\sphinxupquote{param\_err\_comp}}}{\emph{x}, \emph{data}, \emph{err}, \emph{best\_par}, \emph{chi\_sq}, \emph{par}}{}
Function to calculate the errors in the comptonized fit to the spectrum. Returns 1 sigma error bars.
\begin{quote}\begin{description}
\item[{Parameters}] \leavevmode\begin{itemize}
\item {} 
\sphinxstyleliteralstrong{\sphinxupquote{x}} \textendash{} energies of the spectrum in keV

\item {} 
\sphinxstyleliteralstrong{\sphinxupquote{data}} \textendash{} The spectrum data points that are being fitted

\item {} 
\sphinxstyleliteralstrong{\sphinxupquote{err}} \textendash{} the errors in each data point

\item {} 
\sphinxstyleliteralstrong{\sphinxupquote{best\_par}} \textendash{} the best fit parameters of the COMP function to the MCRaT spectrum

\item {} 
\sphinxstyleliteralstrong{\sphinxupquote{chi\_sq}} \textendash{} The value of chi squared of the best fit (value of best\_par)

\item {} 
\sphinxstyleliteralstrong{\sphinxupquote{par}} \textendash{} The COMP spectrum parameter of interest that we would like to find the errors of (‘a’, ‘b’, ‘c’ which
corresponds to the definitions of the COMP function)

\end{itemize}

\item[{Returns}] \leavevmode
returns a (1x2) array of the negative error bar and the positive error bar

\end{description}\end{quote}

\end{fulllineitems}

\index{param\_err\_loop() (in module read\_process\_files)@\spxentry{param\_err\_loop()}\spxextra{in module read\_process\_files}}

\begin{fulllineitems}
\phantomsection\label{\detokenize{read_process_files:read_process_files.param_err_loop}}\pysiglinewithargsret{\sphinxcode{\sphinxupquote{read\_process\_files.}}\sphinxbfcode{\sphinxupquote{param\_err\_loop}}}{\emph{x}, \emph{data}, \emph{err}, \emph{best\_par}, \emph{chi\_sq}, \emph{par}, \emph{d\_par}}{}
Function to calculate the deviation in a Band spectrum parameter such that there is a change of \textasciitilde{}1 in the chi squared
value of the fit.
\begin{quote}\begin{description}
\item[{Parameters}] \leavevmode\begin{itemize}
\item {} 
\sphinxstyleliteralstrong{\sphinxupquote{x}} \textendash{} energies of the spectrum in keV

\item {} 
\sphinxstyleliteralstrong{\sphinxupquote{data}} \textendash{} The spectrum data points that are being fitted

\item {} 
\sphinxstyleliteralstrong{\sphinxupquote{err}} \textendash{} the errors in each data point

\item {} 
\sphinxstyleliteralstrong{\sphinxupquote{best\_par}} \textendash{} the best fit parameters of the Band function to the MCRaT spectrum

\item {} 
\sphinxstyleliteralstrong{\sphinxupquote{chi\_sq}} \textendash{} The value of chi squared of the best fit (value of best\_par)

\item {} 
\sphinxstyleliteralstrong{\sphinxupquote{par}} \textendash{} The Band spectrum parameter of interest that we would like to find the errors of

\item {} 
\sphinxstyleliteralstrong{\sphinxupquote{d\_par}} \textendash{} The initial guess in the change in parameter that would lead to a change in chi squared of \textasciitilde{}1

\end{itemize}

\item[{Returns}] \leavevmode
returns the amount the parameter of interest can change before the fitted chi squared changes by \textasciitilde{}1

\end{description}\end{quote}

\end{fulllineitems}

\index{param\_err\_loop\_comp() (in module read\_process\_files)@\spxentry{param\_err\_loop\_comp()}\spxextra{in module read\_process\_files}}

\begin{fulllineitems}
\phantomsection\label{\detokenize{read_process_files:read_process_files.param_err_loop_comp}}\pysiglinewithargsret{\sphinxcode{\sphinxupquote{read\_process\_files.}}\sphinxbfcode{\sphinxupquote{param\_err\_loop\_comp}}}{\emph{x}, \emph{data}, \emph{err}, \emph{best\_par}, \emph{chi\_sq}, \emph{par}, \emph{d\_par}}{}
Function to calculate the deviation in a COMP spectrum parameter such that there is a change of \textasciitilde{}1 in the chi squared
value of the fit.
\begin{quote}\begin{description}
\item[{Parameters}] \leavevmode\begin{itemize}
\item {} 
\sphinxstyleliteralstrong{\sphinxupquote{x}} \textendash{} energies of the spectrum in keV

\item {} 
\sphinxstyleliteralstrong{\sphinxupquote{data}} \textendash{} The spectrum data points that are being fitted

\item {} 
\sphinxstyleliteralstrong{\sphinxupquote{err}} \textendash{} the errors in each data point

\item {} 
\sphinxstyleliteralstrong{\sphinxupquote{best\_par}} \textendash{} the best fit parameters of the COMP function to the MCRaT spectrum

\item {} 
\sphinxstyleliteralstrong{\sphinxupquote{chi\_sq}} \textendash{} The value of chi squared of the best fit (value of best\_par)

\item {} 
\sphinxstyleliteralstrong{\sphinxupquote{par}} \textendash{} The COMP spectrum parameter of interest that we would like to find the errors of

\item {} 
\sphinxstyleliteralstrong{\sphinxupquote{d\_par}} \textendash{} The initial guess in the change in parameter that would lead to a change in chi squared of \textasciitilde{}1

\end{itemize}

\item[{Returns}] \leavevmode
returns the amount the parameter of interest can change before the fitted chi squared changes by \textasciitilde{}1

\end{description}\end{quote}

\end{fulllineitems}

\index{plot\_golenetskii() (in module read\_process\_files)@\spxentry{plot\_golenetskii()}\spxextra{in module read\_process\_files}}

\begin{fulllineitems}
\phantomsection\label{\detokenize{read_process_files:read_process_files.plot_golenetskii}}\pysiglinewithargsret{\sphinxcode{\sphinxupquote{read\_process\_files.}}\sphinxbfcode{\sphinxupquote{plot\_golenetskii}}}{\emph{sims}, \emph{t\_end}, \emph{delta\_t=1}, \emph{detectable=False}, \emph{save\_plot=False}, \emph{h5=True}}{}
Function to plot the MCRaT results alongside the golenetskii relationship
\begin{quote}\begin{description}
\item[{Parameters}] \leavevmode\begin{itemize}
\item {} 
\sphinxstyleliteralstrong{\sphinxupquote{sims}} \textendash{} An array containing strings of event file base names that will be plotted

\item {} 
\sphinxstyleliteralstrong{\sphinxupquote{t\_end}} \textendash{} An array of times that the light curves of the above sim values end

\item {} 
\sphinxstyleliteralstrong{\sphinxupquote{delta\_t}} \textendash{} The size of the time binds used to get the synthetic GRB light curve/time resolved spectra

\item {} 
\sphinxstyleliteralstrong{\sphinxupquote{detectable}} \textendash{} Switch to plot the luminosities below 10\textasciicircum{}50 ergs/s or not, set to False by default

\item {} 
\sphinxstyleliteralstrong{\sphinxupquote{save\_plot}} \textendash{} Switch to save the plot as a PDF or not

\item {} 
\sphinxstyleliteralstrong{\sphinxupquote{h5}} \textendash{} Switch to specify if the MCRaT output files were HDF5 format or not, default is True. Should be set to
False when an old version of MCRaT was used which outputted the data in text files

\end{itemize}

\item[{Returns}] \leavevmode


\end{description}\end{quote}

\end{fulllineitems}

\index{plotfluidGammaVsTheta() (in module read\_process\_files)@\spxentry{plotfluidGammaVsTheta()}\spxextra{in module read\_process\_files}}

\begin{fulllineitems}
\phantomsection\label{\detokenize{read_process_files:read_process_files.plotfluidGammaVsTheta}}\pysiglinewithargsret{\sphinxcode{\sphinxupquote{read\_process\_files.}}\sphinxbfcode{\sphinxupquote{plotfluidGammaVsTheta}}}{\emph{derivative=False}, \emph{comp\_lundman=False}, \emph{saveplot=False}}{}
This function plots the lorentz factor profile of hydrodynamic outflows, as a function of theta, alongside
Lundman et al.’s (2014b) lorentz factor profile
\begin{quote}\begin{description}
\item[{Parameters}] \leavevmode\begin{itemize}
\item {} 
\sphinxstyleliteralstrong{\sphinxupquote{derivative}} \textendash{} Switch to plot the derivative of the lorentz factor with respect to theta

\item {} 
\sphinxstyleliteralstrong{\sphinxupquote{comp\_lundman}} \textendash{} Switch to compare the hydro simulation to the lorentz factor profile from Lundman et al. (2014b)

\item {} 
\sphinxstyleliteralstrong{\sphinxupquote{saveplot}} \textendash{} Switch to save the plot as a PDF or not

\end{itemize}

\item[{Returns}] \leavevmode


\end{description}\end{quote}

\end{fulllineitems}

\index{polarizationVsAngle() (in module read\_process\_files)@\spxentry{polarizationVsAngle()}\spxextra{in module read\_process\_files}}

\begin{fulllineitems}
\phantomsection\label{\detokenize{read_process_files:read_process_files.polarizationVsAngle}}\pysiglinewithargsret{\sphinxcode{\sphinxupquote{read\_process\_files.}}\sphinxbfcode{\sphinxupquote{polarizationVsAngle}}}{\emph{event\_files}, \emph{time\_start}, \emph{time\_end}, \emph{dt=1}, \emph{phi\_angle=0}, \emph{delta\_theta=1}, \emph{unit='erg/s'}, \emph{plotting=True}, \emph{save\_plot=False}, \emph{dim=2}, \emph{hdf5=True}, \emph{compare\_to\_lund=False}, \emph{p=4}, \emph{thetaj=0.01}, \emph{plot\_lumi=False}}{}
Function to plot how time integrated polarization of the MCRaT simulation changes as a function of observer angle.
There are option to compare the results to those of Lundman et al. (2014b) and to also overplot the peak luminoity
of the light curve at the given angle.
\begin{quote}\begin{description}
\item[{Parameters}] \leavevmode\begin{itemize}
\item {} 
\sphinxstyleliteralstrong{\sphinxupquote{event\_files}} \textendash{} An array of the MCRaT event file base names that are numerically ordered from smallest to largest

\item {} 
\sphinxstyleliteralstrong{\sphinxupquote{time\_start}} \textendash{} An array of the start times of the light curves for the event files specified above

\item {} 
\sphinxstyleliteralstrong{\sphinxupquote{time\_end}} \textendash{} An array of the end times of the light curves for the event files specified above

\item {} 
\sphinxstyleliteralstrong{\sphinxupquote{dt}} \textendash{} If also plotting the peak of the light curve, dt specifies the width of the time bins of the light curves

\item {} 
\sphinxstyleliteralstrong{\sphinxupquote{phi\_angle}} \textendash{} A placeholder variable for future 3D compatability

\item {} 
\sphinxstyleliteralstrong{\sphinxupquote{delta\_theta}} \textendash{} delta theta of the observer viewieng angle for accepting photons, also in degrees.
Should be the same as what was used in the function call to produce the event files passed to this function

\item {} 
\sphinxstyleliteralstrong{\sphinxupquote{unit}} \textendash{} The units of the produced light curve if plotting the light curve peak as well. The default is ‘erg/s’
but can also pass ‘cts/s’

\item {} 
\sphinxstyleliteralstrong{\sphinxupquote{plotting}} \textendash{} Switch, set to True by default, to show the plot produced

\item {} 
\sphinxstyleliteralstrong{\sphinxupquote{save\_plot}} \textendash{} Switch to save the plot as a PDF or not

\item {} 
\sphinxstyleliteralstrong{\sphinxupquote{dim}} \textendash{} Sets the number of dimensions of the MCraT simulation analyzed, should be set to 2 since there is no 3d
compatability yet.

\item {} 
\sphinxstyleliteralstrong{\sphinxupquote{hdf5}} \textendash{} Switch to specify if the MCRaT output files were HDF5 format or not, default is True. Should be set to
False when an old version of MCRaT was used which outputted the data in text files

\item {} 
\sphinxstyleliteralstrong{\sphinxupquote{compare\_to\_lund}} \textendash{} Switch to specify if the user wants to compare their results to those of
Lundman et al. (2014b)

\item {} 
\sphinxstyleliteralstrong{\sphinxupquote{p}} \textendash{} If compare\_to\_lund is set to True, the user needs to specify the proper p value for the lorentz factor
profile. The default is p=4, also accepts p=2.

\item {} 
\sphinxstyleliteralstrong{\sphinxupquote{thetaj}} \textendash{} If compare\_to\_lund is set to True, the user needs to specify the opening angel of the jet in radians,
the default is set to 0.01 rad, also accepts 0.1 rad when p=4.

\item {} 
\sphinxstyleliteralstrong{\sphinxupquote{plot\_lumi}} \textendash{} A switch to plot the peak luminosity of the light curve associated with the passed event\_files

\end{itemize}

\item[{Returns}] \leavevmode
Arrays of the plotted time integrated polarizations and their errors and the polarization angles and their
associated 1 sigma errors

\end{description}\end{quote}

\end{fulllineitems}

\index{read\_flash() (in module read\_process\_files)@\spxentry{read\_flash()}\spxextra{in module read\_process\_files}}

\begin{fulllineitems}
\phantomsection\label{\detokenize{read_process_files:read_process_files.read_flash}}\pysiglinewithargsret{\sphinxcode{\sphinxupquote{read\_process\_files.}}\sphinxbfcode{\sphinxupquote{read\_flash}}}{\emph{fnam}, \emph{length\_scale=1000000000.0}, \emph{make1D=True}}{}
Legacy code that was used to read in FLASH data frames in the old python version of the code, still is used in some
functions comparing MCRaT information to fluid quantities
\begin{quote}\begin{description}
\item[{Parameters}] \leavevmode\begin{itemize}
\item {} 
\sphinxstyleliteralstrong{\sphinxupquote{fnam}} \textendash{} the file name fo the FLASH file, including directory it is located in

\item {} 
\sphinxstyleliteralstrong{\sphinxupquote{make1D}} \textendash{} switch to set the output to be either 1D, if set to True (the default), or keep the returned
information as 2D arrays

\end{itemize}

\item[{Returns}] \leavevmode
returns all of the FLASH file information (x, y, dx, dy, vx, vy, gamma, density, lab density, radius,
theta, pressure) at each hydro node

\end{description}\end{quote}

\end{fulllineitems}

\index{read\_mcrat() (in module read\_process\_files)@\spxentry{read\_mcrat()}\spxextra{in module read\_process\_files}}

\begin{fulllineitems}
\phantomsection\label{\detokenize{read_process_files:read_process_files.read_mcrat}}\pysiglinewithargsret{\sphinxcode{\sphinxupquote{read\_process\_files.}}\sphinxbfcode{\sphinxupquote{read\_mcrat}}}{\emph{file\_name}}{}
Legacy code to read in MCRaT data when it used to output data in text files
\begin{quote}\begin{description}
\item[{Parameters}] \leavevmode
\sphinxstyleliteralstrong{\sphinxupquote{file\_name}} \textendash{} the file name of the MCRaT data file including the directory that it is located

\item[{Returns}] \leavevmode
returns all the data in the given file of interest

\end{description}\end{quote}

\end{fulllineitems}

\index{read\_mcrat\_h5() (in module read\_process\_files)@\spxentry{read\_mcrat\_h5()}\spxextra{in module read\_process\_files}}

\begin{fulllineitems}
\phantomsection\label{\detokenize{read_process_files:read_process_files.read_mcrat_h5}}\pysiglinewithargsret{\sphinxcode{\sphinxupquote{read\_process\_files.}}\sphinxbfcode{\sphinxupquote{read\_mcrat\_h5}}}{\emph{file\_name}, \emph{read\_comv=False}, \emph{read\_stokes=False}, \emph{read\_type=False}}{}
Reads in MCRaT data for current version of MCRaT that outputs data in hdf5 files. Also has support for various
MCRaT switches that can be turned on by the user.
\begin{quote}\begin{description}
\item[{Parameters}] \leavevmode\begin{itemize}
\item {} 
\sphinxstyleliteralstrong{\sphinxupquote{file\_name}} \textendash{} the file name of the MCRaT data file including the directory that it is located

\item {} 
\sphinxstyleliteralstrong{\sphinxupquote{read\_comv}} \textendash{} switch that lets the function know if it should expect/ return comoving 4 momenta data, set to true
if this switch is set to ON in mcrat\_input.h

\item {} 
\sphinxstyleliteralstrong{\sphinxupquote{read\_stokes}} \textendash{} switch that lets the function know if it should expect/ return stokes parameters, set to true
if this switch is set to ON in mcrat\_input.h

\item {} 
\sphinxstyleliteralstrong{\sphinxupquote{read\_type}} \textendash{} switch that lets the function know if it should expect/ return photon type, set to true
if this switch is set to ON in mcrat\_input.h

\end{itemize}

\item[{Returns}] \leavevmode
returns the data read in from the MCRaT data frame

\end{description}\end{quote}

\end{fulllineitems}

\index{wien() (in module read\_process\_files)@\spxentry{wien()}\spxextra{in module read\_process\_files}}

\begin{fulllineitems}
\phantomsection\label{\detokenize{read_process_files:read_process_files.wien}}\pysiglinewithargsret{\sphinxcode{\sphinxupquote{read\_process\_files.}}\sphinxbfcode{\sphinxupquote{wien}}}{\emph{x}, \emph{t}, \emph{d}}{}
Function that produces a Wien spectrum.
\begin{quote}\begin{description}
\item[{Parameters}] \leavevmode\begin{itemize}
\item {} 
\sphinxstyleliteralstrong{\sphinxupquote{x}} \textendash{} Energy in keV

\item {} 
\sphinxstyleliteralstrong{\sphinxupquote{t}} \textendash{} Temp in kelvin

\item {} 
\sphinxstyleliteralstrong{\sphinxupquote{d}} \textendash{} Normalization of the spectrum

\end{itemize}

\item[{Returns}] \leavevmode
returns the Wien spectrum parameterized by some temperature at the energies of interest

\end{description}\end{quote}

\end{fulllineitems}



\chapter{mcrat\_movie.py}
\label{\detokenize{mcrat_movie:module-mcrat_movie}}\label{\detokenize{mcrat_movie:mcrat-movie-py}}\label{\detokenize{mcrat_movie::doc}}\index{mcrat\_movie (module)@\spxentry{mcrat\_movie}\spxextra{module}}
This routine allows the user to make movies of the photons propagating through the hydro simulation.
This routine can also plot the average temperature of the photons vs the average temperature of the matter near the photons

currently this code only works with FLASH hydrodynamic simulations

1st: read FLASH File and MCRaT file

2nd: plot flash temperature and mcrat positions of a bunch of photons

3rd: plot temp vs radius for photons and matter
\index{calc\_spectrum() (in module mcrat\_movie)@\spxentry{calc\_spectrum()}\spxextra{in module mcrat\_movie}}

\begin{fulllineitems}
\phantomsection\label{\detokenize{mcrat_movie:mcrat_movie.calc_spectrum}}\pysiglinewithargsret{\sphinxcode{\sphinxupquote{mcrat\_movie.}}\sphinxbfcode{\sphinxupquote{calc\_spectrum}}}{\emph{p0}, \emph{weight}}{}
produces spectrum in keV
\begin{quote}\begin{description}
\item[{Parameters}] \leavevmode\begin{itemize}
\item {} 
\sphinxstyleliteralstrong{\sphinxupquote{p0}} \textendash{} 

\item {} 
\sphinxstyleliteralstrong{\sphinxupquote{weight}} \textendash{} 

\end{itemize}

\item[{Returns}] \leavevmode


\end{description}\end{quote}

\end{fulllineitems}

\index{follow\_lc\_data() (in module mcrat\_movie)@\spxentry{follow\_lc\_data()}\spxextra{in module mcrat\_movie}}

\begin{fulllineitems}
\phantomsection\label{\detokenize{mcrat_movie:mcrat_movie.follow_lc_data}}\pysiglinewithargsret{\sphinxcode{\sphinxupquote{mcrat\_movie.}}\sphinxbfcode{\sphinxupquote{follow\_lc\_data}}}{\emph{hydro\_sim\_dir}, \emph{hydro\_sim\_name}, \emph{last\_frame}, \emph{mcrat\_sim\_data\_dir}, \emph{maxx}, \emph{miny}, \emph{maxy}, \emph{theta}, \emph{time}, \emph{dt}, \emph{fps}, \emph{num\_photon\_plot=10000000.0}, \emph{read\_type=False}, \emph{energy\_range=None}}{}~\begin{quote}\begin{description}
\item[{Parameters}] \leavevmode\begin{itemize}
\item {} 
\sphinxstyleliteralstrong{\sphinxupquote{hydro\_sim\_dir}} \textendash{} 

\item {} 
\sphinxstyleliteralstrong{\sphinxupquote{hydro\_sim\_name}} \textendash{} 

\item {} 
\sphinxstyleliteralstrong{\sphinxupquote{last\_frame}} \textendash{} 

\item {} 
\sphinxstyleliteralstrong{\sphinxupquote{mcrat\_sim\_data\_dir}} \textendash{} 

\item {} 
\sphinxstyleliteralstrong{\sphinxupquote{maxx}} \textendash{} in real units, not code units

\item {} 
\sphinxstyleliteralstrong{\sphinxupquote{miny}} \textendash{} 

\item {} 
\sphinxstyleliteralstrong{\sphinxupquote{maxy}} \textendash{} 

\item {} 
\sphinxstyleliteralstrong{\sphinxupquote{theta}} \textendash{} in degrees

\item {} 
\sphinxstyleliteralstrong{\sphinxupquote{time}} \textendash{} 

\item {} 
\sphinxstyleliteralstrong{\sphinxupquote{dt}} \textendash{} 

\item {} 
\sphinxstyleliteralstrong{\sphinxupquote{fps}} \textendash{} 

\item {} 
\sphinxstyleliteralstrong{\sphinxupquote{num\_photon\_plot}} \textendash{} 

\item {} 
\sphinxstyleliteralstrong{\sphinxupquote{read\_type}} \textendash{} 

\end{itemize}

\item[{Returns}] \leavevmode


\end{description}\end{quote}

\end{fulllineitems}

\index{get\_flash\_data() (in module mcrat\_movie)@\spxentry{get\_flash\_data()}\spxextra{in module mcrat\_movie}}

\begin{fulllineitems}
\phantomsection\label{\detokenize{mcrat_movie:mcrat_movie.get_flash_data}}\pysiglinewithargsret{\sphinxcode{\sphinxupquote{mcrat\_movie.}}\sphinxbfcode{\sphinxupquote{get\_flash\_data}}}{\emph{file\_num}, \emph{f}, \emph{max\_x}, \emph{min\_y}, \emph{max\_y}, \emph{flash\_l\_scale}}{}
This function collects all the data from a given FLASH file and prepares it to be plotted with imshow.
\begin{quote}\begin{description}
\item[{Parameters}] \leavevmode\begin{itemize}
\item {} 
\sphinxstyleliteralstrong{\sphinxupquote{file\_num}} \textendash{} The frame number

\item {} 
\sphinxstyleliteralstrong{\sphinxupquote{f}} \textendash{} The path to the file as well as the FLASH file name preceeding the file number

\item {} 
\sphinxstyleliteralstrong{\sphinxupquote{max\_x}} \textendash{} The maximum x that should be plotted (the input should be in physical units, not in code units)

\item {} 
\sphinxstyleliteralstrong{\sphinxupquote{min\_y}} \textendash{} The minimum y of the portion of the hydro frame that will be plotted (the input should be in code units)

\item {} 
\sphinxstyleliteralstrong{\sphinxupquote{max\_y}} \textendash{} The maximum y of the portion of the hydro frame that will be plotted (the input should be in code units)

\end{itemize}

\item[{Returns}] \leavevmode
returns a 2D array of the density for plotting with imshow

\end{description}\end{quote}

\end{fulllineitems}

\index{get\_indexes\_data() (in module mcrat\_movie)@\spxentry{get\_indexes\_data()}\spxextra{in module mcrat\_movie}}

\begin{fulllineitems}
\phantomsection\label{\detokenize{mcrat_movie:mcrat_movie.get_indexes_data}}\pysiglinewithargsret{\sphinxcode{\sphinxupquote{mcrat\_movie.}}\sphinxbfcode{\sphinxupquote{get\_indexes\_data}}}{\emph{mcrat\_f}, \emph{num}, \emph{ph\_num}, \emph{angle}, \emph{t}, \emph{dt}, \emph{fps}, \emph{read\_comv=False}, \emph{read\_stokes=False}, \emph{read\_type=False}, \emph{dtheta\_deg=1}, \emph{energy\_range=None}}{}
This function gets the set of indexes of the photons that will be plotted in the imshow window as well as the
indexes of all the photons in order to calculate the spectrum
\begin{quote}\begin{description}
\item[{Parameters}] \leavevmode\begin{itemize}
\item {} 
\sphinxstyleliteralstrong{\sphinxupquote{mcrat\_f}} \textendash{} 

\item {} 
\sphinxstyleliteralstrong{\sphinxupquote{num}} \textendash{} 

\item {} 
\sphinxstyleliteralstrong{\sphinxupquote{ph\_num}} \textendash{} 

\item {} 
\sphinxstyleliteralstrong{\sphinxupquote{angle}} \textendash{} 

\item {} 
\sphinxstyleliteralstrong{\sphinxupquote{t}} \textendash{} 

\item {} 
\sphinxstyleliteralstrong{\sphinxupquote{dt}} \textendash{} 

\item {} 
\sphinxstyleliteralstrong{\sphinxupquote{fps}} \textendash{} 

\item {} 
\sphinxstyleliteralstrong{\sphinxupquote{read\_comv}} \textendash{} 

\item {} 
\sphinxstyleliteralstrong{\sphinxupquote{read\_stokes}} \textendash{} 

\item {} 
\sphinxstyleliteralstrong{\sphinxupquote{dtheta\_deg}} \textendash{} 

\end{itemize}

\item[{Returns}] \leavevmode


\end{description}\end{quote}

\end{fulllineitems}

\index{get\_plot\_data() (in module mcrat\_movie)@\spxentry{get\_plot\_data()}\spxextra{in module mcrat\_movie}}

\begin{fulllineitems}
\phantomsection\label{\detokenize{mcrat_movie:mcrat_movie.get_plot_data}}\pysiglinewithargsret{\sphinxcode{\sphinxupquote{mcrat\_movie.}}\sphinxbfcode{\sphinxupquote{get\_plot\_data}}}{\emph{frame\_number}, \emph{flash\_file}, \emph{mcrat\_file}, \emph{t\_ph=None}, \emph{t\_f=None}, \emph{plot\_temp\_curves=False}, \emph{read\_comv=False}, \emph{read\_stokes=False}, \emph{read\_type=False}, \emph{dtheta\_deg=1}}{}~\begin{quote}\begin{description}
\item[{Parameters}] \leavevmode\begin{itemize}
\item {} 
\sphinxstyleliteralstrong{\sphinxupquote{frame\_number}} \textendash{} 

\item {} 
\sphinxstyleliteralstrong{\sphinxupquote{flash\_file}} \textendash{} 

\item {} 
\sphinxstyleliteralstrong{\sphinxupquote{mcrat\_file}} \textendash{} 

\item {} 
\sphinxstyleliteralstrong{\sphinxupquote{t\_ph}} \textendash{} 

\item {} 
\sphinxstyleliteralstrong{\sphinxupquote{t\_f}} \textendash{} 

\item {} 
\sphinxstyleliteralstrong{\sphinxupquote{plot\_temp\_curves}} \textendash{} 

\item {} 
\sphinxstyleliteralstrong{\sphinxupquote{read\_comv}} \textendash{} 

\item {} 
\sphinxstyleliteralstrong{\sphinxupquote{read\_stokes}} \textendash{} 

\item {} 
\sphinxstyleliteralstrong{\sphinxupquote{read\_type}} \textendash{} 

\item {} 
\sphinxstyleliteralstrong{\sphinxupquote{dtheta\_deg}} \textendash{} 

\end{itemize}

\item[{Returns}] \leavevmode


\end{description}\end{quote}

\end{fulllineitems}



\chapter{Indices and tables}
\label{\detokenize{index:indices-and-tables}}\begin{itemize}
\item {} 
\DUrole{xref,std,std-ref}{genindex}

\item {} 
\DUrole{xref,std,std-ref}{modindex}

\item {} 
\DUrole{xref,std,std-ref}{search}

\end{itemize}


\renewcommand{\indexname}{Python Module Index}
\begin{sphinxtheindex}
\let\bigletter\sphinxstyleindexlettergroup
\bigletter{m}
\item\relax\sphinxstyleindexentry{mclib}\sphinxstyleindexpageref{mclib:\detokenize{module-mclib}}
\item\relax\sphinxstyleindexentry{mcrat\_movie}\sphinxstyleindexpageref{mcrat_movie:\detokenize{module-mcrat_movie}}
\indexspace
\bigletter{r}
\item\relax\sphinxstyleindexentry{read\_process\_files}\sphinxstyleindexpageref{read_process_files:\detokenize{module-read_process_files}}
\end{sphinxtheindex}

\renewcommand{\indexname}{Index}
\printindex
\end{document}